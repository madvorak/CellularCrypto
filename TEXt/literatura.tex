%%% Seznam pou�it� literatury (bibliografie)
%%%
%%% Pro vytv��en� bibliografie pou��v�me bibTeX. Ten zpracov�v�
%%% citace v textu (nap�. makro \cite{...}) a vyhled�v� k nim literaturu
%%% v souboru literatura.bib.
%%%
%%% P��kaz \bibliographystyle ur�uje, jak�m stylem budou citov�ny odkazy
%%% v textu. V z�vorce je n�zev zvolen�ho souboru .bst. Styly plainnat
%%% a unsrt jsou standardn� sou��st� latexov�ch distribuc�. Styl czplainnat
%%% je dod�v�n s touto �ablonou a bibTeX ho hled� v aktu�ln�m adres��i.

\bibliographystyle{czplainnat}    %% Autor (rok) s �esk�mi spojkami
% \bibliographystyle{plainnat}    %% Autor (rok) s anglick�mi spojkami
% \bibliographystyle{unsrt}       %% [��slo]

\renewcommand{\bibname}{Seznam pou�it� literatury}

%%% Vytvo�en� seznamu literatury. Pozor, pokud jste necitovali ani jednu
%%% polo�ku, seznam se automaticky vynech�.

\bibliography{literatura}

http://mathoverflow.net/questions/128903/expected-edit-distance?rq=1

http://math.stackexchange.com/questions/375505/what-is-the-average-levenshtein-distance-between-two-random-binary-strings-of-le

http://psoup.math.wisc.edu/mcell/rullex_life.html